\documentclass{Ximera}
\title{This is the title}
\author{Me}


\begin{document}
\maketitle
\begin{abstract}
This is a test of fold commands.
\end{abstract}


To use fold there are now several options if you use the command 
\begin{verbatim}
\flexFold{#1}{#2}
\end{verbatim}

First, there are user local commands for each fold environment being used, which is the first argument. You want to assign it to a number 0, 1, or 2.

If you assign it a 0, it will be folded, like so:
\begin{verbatim}
\flexFold{0}{This is some content. There is a lot of content like it, but this one is mine.}
\end{verbatim}

Creates:
\flexFold{0}{This is some content. There is a lot of content like it, but this one is mine.}

If you use a 1, it will start expanded, like so:

\begin{verbatim}
\flexFold{1}{This is some content. There is a lot of content like it, but this one is mine.}
\end{verbatim}

Creates:
\flexFold{1}{This is some content. There is a lot of content like it, but this one is mine.}

And finally, if you use 2, then the content will be expanded, but without the arrow button in the top right, like so:
\begin{verbatim}
\flexFold{2}{This is some content. There is a lot of content like it, but this one is mine.}
\end{verbatim}

Creates:
\flexFold{2}{This is some content. There is a lot of content like it, but this one is mine.}

In addition to these commands we now have global commands,
\begin{verbatim}
\collapseAllFolds
\expandAllFolds
\expandAllFoldsPDF
\end{verbatim}

These work as you would suspect, and they overwrite the local variable assigned when the environment itself is called. Thus:

\begin{verbatim}
\collapseAllFolds
\flexFold{0}{This is some content. There is a lot of content like it, but this one is mine.}
\flexFold{1}{This is some content. There is a lot of content like it, but this one is mine.}
\flexFold{2}{This is some content. There is a lot of content like it, but this one is mine.}
\end{verbatim}

Creates:
\collapseAllFolds
\flexFold{0}{This is some content. There is a lot of content like it, but this one is mine.}
\flexFold{1}{This is some content. There is a lot of content like it, but this one is mine.}
\flexFold{2}{This is some content. There is a lot of content like it, but this one is mine.}

and 
\begin{verbatim}
\expandAllFolds
\flexFold{0}{This is some content. There is a lot of content like it, but this one is mine.}
\flexFold{1}{This is some content. There is a lot of content like it, but this one is mine.}
\flexFold{2}{This is some content. There is a lot of content like it, but this one is mine.}
\end{verbatim}

Creates:
\expandAllFolds
\flexFold{0}{This is some content. There is a lot of content like it, but this one is mine.}
\flexFold{1}{This is some content. There is a lot of content like it, but this one is mine.}
\flexFold{2}{This is some content. There is a lot of content like it, but this one is mine.}

and:

\begin{verbatim}
\expandAllFoldsPDF
\flexFold{0}{This is some content. There is a lot of content like it, but this one is mine.}
\flexFold{1}{This is some content. There is a lot of content like it, but this one is mine.}
\flexFold{2}{This is some content. There is a lot of content like it, but this one is mine.}
\end{verbatim}

Creates:
\expandAllFoldsPDF
\flexFold{0}{This is some content. There is a lot of content like it, but this one is mine.}
\flexFold{1}{This is some content. There is a lot of content like it, but this one is mine.}
\flexFold{2}{This is some content. There is a lot of content like it, but this one is mine.}


\end{document}
